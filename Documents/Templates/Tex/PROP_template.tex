\documentclass[letterpaper,12pt]{article}
\usepackage{graphicx}
\usepackage{geometry}
\usepackage{fancyhdr}
\usepackage{lastpage}
\usepackage{hyperref}

\geometry{margin=1.5in, footskip=1in}

\pagestyle{fancy}
\fancyhead{}
\renewcommand{\headrulewidth}{0pt} 
\fancyfoot{}
\fancyfoot[R]{\thepage/\pageref{LastPage}}
\fancyfoot[C]{TAMU CSCE 483 Project Proposal}

\title{Project title\\{\Large Project Proposal}}
\author{\begin{Large} Author 1\\Author 2\\Author 3\\Author 4\\ \end{Large} Department of Computer Science and Engineering\\Texas A\&M University\\[4ex]}
\date{\today}

\makeatletter         
\def\@maketitle{
\centering \includegraphics[width = 2in]{images/holder.jpg}\\[8ex]
\begin{center}
{\Huge \bfseries \sffamily \@title }\\[4ex] 
{\LARGE  \@author} 
{\Large \@date\\[8ex]}
\end{center}}
\makeatother



\begin{document}

\maketitle
\thispagestyle{fancy}
\newpage

\tableofcontents
\newpage

\section*{Executive Summary (1 page; 5 points)}
The executive summary is a brief description of the project. The
purpose is to give a quick overview of (1) need, goal and objectives,
(2) design and implementation, and (3) expected results and benefits
of the project. The intended audience of the executive summary is a
program director, someone who makes decisions about which projects will
receive funding. Since the executive summary is a summary, it should be
written last.

\section{Introduction (1 page)}
\subsection{Problem Background (5 points)}
Describe the general scope of your project, and the specific problem
that your project is addressing. Go from general (e.g., search and
rescue robotics) to specific (e.g., GPS navigation). In most of the
senior design projects, the problem background is an area of research
or an application domain within computer engineering.

\subsection{Needs statement (5 points)} After you have introduced the
problem domain, it is time to define a specific need that your project
will address. Articulate the need as an expression of dissatisfaction
with the current situation.

\subsection{Goal and objectives (5 points)}

The goal is a brief, general, and ideal response to the needs
statement. The need describes the current, unsatisfactory situation;
the goal describes the future condition to which we aspire. The goal
statement is so ideal that it would be difficult to decide when it
was achieved. It rather establishes a general direction for the design
mission. In contrast, the objectives (there will likely be more than
one) are quantifiable expectations of performance. The objectives should
also include a description of the conditions under which a design must
perform. Specifying the operating conditions will allow you to evaluate
the performance of different design options under comparable conditions.

\subsection{Design constraints and feasibility (5 points)} Describe
the constraints (e.g., technical, physical, economical, temporal)
that you have to work with. In many cases, your needs statement will
have already identified some of the constraints that your design will
have to meet. Assess the extent to which your project objectives can
be accomplished.

\section{Literature and technical survey (1-2 pages; 10 points)} Describe
prior research and development efforts that are specifically related to
your problem, your needs statement, and your goals and objectives. This is
not meant to be a comprehensive survey of an engineering discipline, but a
concise overview of the most significant results that are tightly related
to your project. As a guideline, this section should include a review
of not less than five commercial products or research projects. This
review should conclude with a statement that explains the extent to which
your proposed design relates to these other products/projects (e.g., is
it better? or faster? or cheaper? does it target a new niche market?).
\begin{itemize} 
\item Existing product or project 1 
\item Existing product or project 2 
\item Existing product or project 3 
\item Existing product or project 4 
\item Existing product or project 5 
\end{itemize}

\section{Proposed Work}
\subsection{Evaluation of alternative solutions (1 page, 15 points)}
This is a critical aspect of your proposal. For any goal there are likely
many alternative solutions. In most cases, the alternative solutions
will emerge from your literature and technical survey. What you have
to do here is analyze the pros and cons of each of these solutions
(and hopefully additional solutions you come up with), and justify your
decision to opt for a particular solution. As a guideline, this section
should include not less than five alternative solutions:
\begin{itemize}
\item Alternative solution 1
\item Alternative solution 2
\item Alternative solution 3
\item Alternative solution 4
\item Alternative solution 5
\end{itemize}

\subsection{Design specifications (3-4 pages, 15 points)}
Once you have identified a solution that addresses the needs of the
project, it is time to present the specifics of your design. Start with
a high-level block diagram of the system (i.e., 5 building blocks or
modules), followed by a description of each module. This description
should include techniques (e.g., algorithms, devices), parts (e.g.,
hardware, software), and the “glue logic” that will make the system
work.  Your proposed design should build support for the feasibility of
your project.

\subsection{Approach for design validation (1 page, 10 points)}
This is a very simple but important aspect of your project. How will you
test that your system does what it was designed to do? Does it solve the
stated need? The validation tests should be consistent with the conditions
under which your design must perform, as stated in the project objectives.


\section{Engineering standards}
\subsection{Project management (1 page, 10 points)}
Briefly list the qualifications of the team members and decide who will
be in charge of each of the different areas in the project (team leader,
systems design, software design, hardware design, finance and purchases,
testing, technical reporting, etc.) Describe the mechanisms that will
be used to manage the project as a team (e.g., brainstorming sessions,
tracking progress). A regular work schedule should be included here,
showing the times when the team will meet in the lab to work on the
project (in addition to meeting times).  

NOTE. Each person in the team
should have technical responsibilities. For instance, it would not be
acceptable that one person handles only purchasing and documentation
but does not perform any development work.

\subsection{Schedule of tasks, Pert and Gantt charts (1 page, 5 points)}
Break down the project into clearly identified sub-tasks, analyze
dependencies among them, identify critical paths, and design a feasible
schedule for accomplishing these tasks.

\subsection{Economic analysis (1/2 page; 2 points)}
Some economic issues were already considered during the analysis of
constraints and the itemized budget. Here you provide a further economic
analysis, were your system to become a commercial product:
\begin{itemize}
\item Economical viability: potential marketability of the system,
expected volume production costs (as opposed to prototyping costs)
\item Sustainability: are system parts available from more than
one vendor? What maintenance and support will the product require?
\item Manufacturability: what is the effect of component tolerances on
system performance?, worst- case analysis, expected production yield,
testability and compliance to regulations (e.g., FCC)
\end{itemize}

\subsection{Societal, safety and environmental analysis (1/2 page; 2 points)}
What is the potential impact of your project to society, both beneficial
(e.g., quality of life), and detrimental (e.g., loss of privacy)? What are
the safety precautions that you have to take when working on the project
(e.g., personal injuries, damage to the equipment/facilities)? What is
the potential environmental impact of your project (e.g., pollution)
and how will you minimize it?

\subsection{Itemized budget (1 page; 3 points)}
Detailed budget of all costs expected to be incurred during the project
(e.g., parts, fabrication services).

\section{References (2 points)}
Here you acknowledge all the documents that you used as references
throughout the text. Please use the IEEE reference format (
\url{http://www.ieee.org/documents/ieeecitationref.pdf} )

\section{Appendices (1 point)}
\subsection{Product datasheets}
Include product datasheets that may be particularly relevant to your
proposed work. Say you want to use a certain type of microcontroller,
because it has just the right combination features (e.g., types of I/O
ports, or power consumption, etc). You would then attach datasheets for
this product.  NOTE: including a data sheet does not replace the need
for explaining how the component works and how it will be integrated in
the system (section 4.2 ).

\section{Bios and CVs}
Include a brief bio-sketch and CV for each team member. The bio-sketch
is a brief summary of your professional / educational accomplishments,
whereas the CV is a more comprehensive and detailed (itemized) description
of you qualifications. Browse through some of the IEEE Transactions (
\url{http://ieeexplore.ieee.org/} ) for examples of typical bio-sketches.



\end{document}
